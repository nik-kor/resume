%-----------------------------------------
% Korotkih Nikita
% Resumé
%
% URL: https://github.com/nik-kor/resume
%-----------------------------------------

\documentclass[11pt,a4paper,sans]{moderncv}
\moderncvstyle{classic}
\moderncvcolor{green}

% adjust the page margins
\usepackage[scale=0.85]{geometry}

% FONTS
\usepackage[utf8]{inputenc}
\usepackage[russian]{babel}

% personal data
\name{Никита}{Коротких}
\phone[mobile]{+7~(910)~474~69~89}
\email{nikita.korotkih@gmail.com}
\social[github]{nik-kor}
\photo[64pt][0.4pt]{picture}
\extrainfo{Дата рождения: 06/10/1986}

\begin{document}

\makecvtitle

\section{Образование}
\cventry{2004 - 2010}{Специалист}{МАИ(НИУ)}
    {\newline{Специальность: Авиационные приборы и измерительно-вычислительные комплексы}}
    {\newline{Тема диплома: Разработка средств для дистанционного диагностирования приборов и систем}}{}

\section{Навыки}
\cvitem{Языки}{
    \begin{itemize}
        \item Javascript
            \begin{itemize}
                \item native
                \item широкий кругозор и практический опыт использования различных библиотек и фреймворков
                (node.js, Angular.js, Backbone.js, Jasmine, Mocha, jquery, q, npm, bower, require.js, grunt, lineman.js, etc)
            \end{itemize}
        \item PHP - уверенное владение языком
        \item Python - базовое знание. Понимаю питонский код. Могу фиксить баги
    \end{itemize}
}
\cvitem{Проектиро-
вание}{
    \begin{itemize}
        \item знание принципов ООП(SOLID) и практическое применение паттернов
        \item проектирование интерфейсов сервисов с использованием Thrift IDL
        \item опыт проектирования RESTful API
    \end{itemize}
}
\cvitem{Тестиро-
вание}{
    \begin{itemize}
        \item опыт написание юнит тестов - PHPUnit, PyUnit, Jasmine, Karma.js
        \item опыт написание функциональных тестов - Selenium, Phantom
        \item Specification By Example methodology
    \end{itemize}
}
\cvitem{Верстка}{
    \begin{itemize}
        \item HTML/CSS
        \item Less
        \item БЭМ
    \end{itemize}
}
\cvitem{Базы данных}{
    \begin{itemize}
        \item SQL - знаком с теорией(читал книжку Дж. Дейта). Умею писать запросы. Представляю, что такое проектирования схемы БД.
        \item MySQL - базовые знания. Умею писать запросы, но с большими объемами данных не работал, оптимизацией не занимался.
    \end{itemize}
}
\cvitem{Системное администрировние}{
    \begin{itemize}
        \item Linux - рабочая станция. Есть навыки администрирования
        \item bash, awk, sed - использую регулярно
        \item Docker - в последнее время очень интересуюсь данным проектом
    \end{itemize}
}
\cvitem{CVS}{
    \begin{itemize}
        \item git - уверенное знание
        \item svn - если вдруг надо будет - вспомню)
    \end{itemize}
}
\cvitem{ Практики}{
    \begin{itemize}
        \item scrum-мастер. Постигаю скрам на практике. Успел наступить на много грабель. Типа знаю как "правильно"
        \item kanban - знаю в теории
        \item есть разный опыт участие в командой разработки как в качестве обычного разработчики, так и в качестве тим-лида/наставника
        \item верю в успешные горизонтальные scrum-команды без явно выраженного тим-лида
        \item Code review как обязательная часть рабочего процесса
    \end{itemize}
}

\section{Опыт}
\cventry{2012--настоящее время}{\url{http://www.gdeetotdom.ru}}{Технолог}{}{}{
    \begin{itemize}
         \item Разработка и поддержка frontend направления компании
         \item scrum-master. Вцелом продвижение в компании идей гибкой разработки
         \item Наставничество, участие в обсуждениях и помощь при проектировании / реализации проектов компании.
         \item формирование команд, участие в поисках новых сотрудников
    \end{itemize}
}
\cventry{2011--2012}{\url{http://www.gdeetotdom.ru}}{Ведущий разработчик}{}{}{
    \begin{itemize}
        \item ведение команды разработчиков(размер команды около 4-х человек)
        \item разработал единый подход к написанию клиентских приложений в личном кабинете
        \item участие в реализации проектов в качестве разработчика
    \end{itemize}
}
\cventry{2010--2011}{\url{http://www.gdeetotdom.ru}}{Старший разработчик}{}{}{
    \begin{itemize}
        \item реализации проектов компании
    \end{itemize}
}
\cventry{2008--2010}{\url{http://www.aft.ru}}{Разработчик}{}{}{
    \begin{itemize}
        \item Разработка, поддержка сайтов. PHP, JS, системное администрирование.
    \end{itemize}
}
\cventry{2007-2008 (полгода)}{\url{http://www.qsoft.ru}}{Разработчик}{}{}{
    \begin{itemize}
        \item Сайты. PHP
    \end{itemize}
}
\cventry{2007-2007 (полгода)}{\url{http://www.rusite.ru}}{Разработчик}{}{}{
    \begin{itemize}
        \item Сайты. PHP
    \end{itemize}
}
\cventry{2006-2007}{\url{http://www.aviatex.ru}}{Техник}{}{}{
    \begin{itemize}
        \item Схемотехника и программирование микроконтроллеров. Atmel AVR, ПАИС, P-CAD
    \end{itemize}
}

\section{Разное}
    \cvlistitem{\url{https://github.com/nik-kor}}
    \cvlistitem{\url{https://www.facebook.com/nikita.korotkih}}
    \cvlistitem{\url{http://vk.com/id17392292}}
    \cvlistitem{\url{https://twitter.com/NikitaKorotkih}}
    \cvlistitem{Увлекаюсь альпинизмом и скалолазанием}
    \cvlistitem{Английский язык - технический.}

\end{document}
